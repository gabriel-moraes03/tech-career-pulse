\documentclass[12pt,a4paper]{article}
\usepackage[utf8]{inputenc}
\usepackage[brazil]{babel}
\usepackage{geometry}
\usepackage{graphicx}
\usepackage{xcolor}
\usepackage{listings}
\usepackage{hyperref}
\usepackage{enumitem}
\usepackage{tikz}
\usetikzlibrary{shapes,arrows,positioning,fit,backgrounds}

\geometry{margin=2.5cm}

\definecolor{primarycolor}{RGB}{41, 128, 185}
\definecolor{secondarycolor}{RGB}{52, 152, 219}
\definecolor{accentcolor}{RGB}{231, 76, 60}

\title{\textbf{Tech Career Pulse} \\
\large Arquitetura de uma Plataforma de Inteligência de Mercado Tech \\
com IA Generativa}
\author{Gabriel Moraes}
\date{\today}

\begin{document}

\maketitle
\tableofcontents
\newpage

%=============================================================================
\section{Visão Geral do Projeto}
%=============================================================================

\subsection{Problema a Resolver}

O mercado de tecnologia é extremamente dinâmico, com novas ferramentas, linguagens e frameworks surgindo constantemente. Profissionais de tech enfrentam dificuldades em:

\begin{itemize}
    \item \textbf{Identificar tendências de mercado}: Quais tecnologias estão em alta demanda?
    \item \textbf{Planejar carreira}: Qual caminho seguir para alcançar um objetivo profissional?
    \item \textbf{Encontrar vagas compatíveis}: Vagas que realmente fazem sentido para o perfil atual
    \item \textbf{Desenvolver portfólio}: Quais projetos construir para se destacar?
\end{itemize}

\subsection{Solução Proposta}

\textbf{Tech Career Pulse} é uma plataforma de inteligência de mercado que:

\begin{enumerate}
    \item \textbf{Coleta} automaticamente milhares de vagas de emprego através de web scraping
    \item \textbf{Processa} essas vagas extraindo informações estruturadas (skills, senioridade, localização)
    \item \textbf{Analisa} tendências de mercado e gera insights visuais
    \item \textbf{Utiliza IA Generativa} para criar experiências personalizadas:
    \begin{itemize}
        \item Match inteligente entre perfil do usuário e vagas
        \item Roadmaps de carreira personalizados
        \item Sugestões de projetos para portfólio
    \end{itemize}
\end{enumerate}

\subsection{Diferencial Competitivo}

\begin{itemize}
    \item \textbf{Dados reais}: Não é apenas um chatbot genérico, usa dados reais do mercado
    \item \textbf{IA contextual}: IA integrada em workflows específicos, não chat genérico
    \item \textbf{Foco em tech}: Especializado no mercado de tecnologia brasileiro
    \item \textbf{Open source}: Demonstra habilidades técnicas completas
\end{itemize}

%=============================================================================
\section{Arquitetura da Plataforma}
%=============================================================================

\subsection{Visão de Alto Nível}

A plataforma segue arquitetura de 3 camadas com pipeline de dados isolado:

\begin{verbatim}
┌─────────────────────────────────────────────────────────┐
│                     FRONTEND (React)                     │
│  - Dashboard de Insights                                 │
│  - Página de Vagas com Match IA                         │
│  - Página de Roadmap Personalizado                      │
│  - Página de Projetos para Portfólio                    │
└────────────────────┬────────────────────────────────────┘
                     │ HTTP/REST
                     ▼
┌─────────────────────────────────────────────────────────┐
│              BACKEND (Spring Boot 3.5.7)                │
│  ┌───────────────┐  ┌──────────────┐  ┌──────────────┐ │
│  │   Insights    │  │   Match IA   │  │   Roadmap    │ │
│  │   Service     │  │   Service    │  │   Service    │ │
│  └───────────────┘  └──────────────┘  └──────────────┘ │
│  ┌───────────────────────────────────────────────────┐ │
│  │          IA Orchestration Service                 │ │
│  │  - Cache inteligente (hash-based)                 │ │
│  │  - Integração com Gemini/OpenAI                   │ │
│  │  - Rate limiting                                   │ │
│  └───────────────────────────────────────────────────┘ │
└────────────────────┬────────────────────────────────────┘
                     │
                     ▼
┌─────────────────────────────────────────────────────────┐
│                  CAMADA DE DADOS                        │
│  ┌──────────────┐      ┌──────────────┐                │
│  │  PostgreSQL  │      │    Redis     │                │
│  │  - Vagas     │      │  - Cache BI  │                │
│  │  - Skills    │      │  - Cache IA  │                │
│  │  - Interações│      │              │                │
│  └──────────────┘      └──────────────┘                │
└─────────────────────────────────────────────────────────┘
                     ▲
                     │
┌─────────────────────────────────────────────────────────┐
│              DATA PIPELINE (Python)                     │
│  ┌──────────┐   ┌──────────┐   ┌────────────┐         │
│  │ Scraper  │ → │ Processor│ → │  Enricher  │         │
│  │(Playwright)│  │  (NLP)   │   │  (Skills)  │         │
│  └──────────┘   └──────────┘   └────────────┘         │
│                                                         │
│  Scheduler: Cron/Airflow (executa diariamente)        │
└─────────────────────────────────────────────────────────┘
                     ▲
                     │
              [ Gupy.io, LinkedIn, Indeed... ]
\end{verbatim}

\subsection{Componentes Principais}

\subsubsection{Data Pipeline (Python)}

\textbf{Responsabilidades:}
\begin{itemize}
    \item Coletar vagas de múltiplas fontes
    \item Extrair informações estruturadas
    \item Detectar skills usando NLP/Pattern Matching
    \item Inferir senioridade
    \item Persistir dados normalizados
\end{itemize}

\textbf{Tecnologias:}
\begin{itemize}
    \item Playwright (web scraping)
    \item psycopg2 (PostgreSQL)
    \item unidecode (normalização de texto)
    \item Agendamento: Cron ou Apache Airflow
\end{itemize}

\subsubsection{Backend API (Spring Boot)}

\textbf{Responsabilidades:}
\begin{itemize}
    \item Expor APIs REST para frontend
    \item Implementar lógica de negócio
    \item Orquestrar chamadas para IA
    \item Gerenciar cache (Redis)
    \item Aplicar rate limiting
\end{itemize}

\textbf{Tecnologias:}
\begin{itemize}
    \item Spring Boot 3.5.7
    \item Spring Data JPA
    \item Spring Cache (Redis)
    \item WebClient (chamadas HTTP para IA)
    \item Flyway (migrations)
\end{itemize}

\subsubsection{Frontend (React)}

\textbf{Responsabilidades:}
\begin{itemize}
    \item Interface responsiva e intuitiva
    \item Visualizações de dados (gráficos)
    \item Formulários para interação com IA
    \item Gestão de estado
\end{itemize}

\textbf{Tecnologias:}
\begin{itemize}
    \item React 18
    \item React Router (navegação)
    \item Axios (HTTP client)
    \item Recharts ou D3.js (visualizações)
    \item Tailwind CSS ou Material-UI (estilo)
\end{itemize}

\subsubsection{IA Generativa (Gemini/OpenAI)}

\textbf{Responsabilidades:}
\begin{itemize}
    \item Gerar roadmaps personalizados
    \item Calcular match entre perfil e vagas
    \item Sugerir projetos de portfólio
    \item Análise de gaps de skills
\end{itemize}

\textbf{Integração:}
\begin{itemize}
    \item Google Gemini API (recomendado - free tier generoso)
    \item Alternativa: OpenAI GPT-4o-mini
    \item Cache obrigatório (hash de inputs)
\end{itemize}

%=============================================================================
\section{Funcionalidades Detalhadas}
%=============================================================================

\subsection{Feature 1: Dashboard de Insights}

\textbf{Status:} \textcolor{green}{JÁ IMPLEMENTADO}

\textbf{Objetivo:} Apresentar análises do mercado de trabalho tech em tempo real.

\textbf{Visualizações:}
\begin{enumerate}
    \item \textbf{Top 10 Tecnologias Mais Demandadas}
    \begin{itemize}
        \item Tipo: Gráfico de barras horizontal
        \item Dados: COUNT de vagas por skill
        \item Insight: "Java está em 1.234 vagas (23\% do total)"
    \end{itemize}

    \item \textbf{Distribuição por Senioridade}
    \begin{itemize}
        \item Tipo: Gráfico de pizza
        \item Dados: Júnior, Pleno, Sênior, Especialista
        \item Insight: "60\% das vagas são para Pleno ou acima"
    \end{itemize}

    \item \textbf{Modelo de Trabalho}
    \begin{itemize}
        \item Tipo: Gráfico de barras
        \item Dados: Presencial, Híbrido, Remoto
        \item Insight: "45\% das vagas são remotas"
    \end{itemize}

    \item \textbf{Top 10 Cidades}
    \begin{itemize}
        \item Tipo: Lista ranqueada
        \item Dados: Localização + COUNT
        \item Insight: "São Paulo lidera com 2.345 vagas"
    \end{itemize}

    \item \textbf{Áreas Mais Quentes}
    \begin{itemize}
        \item Tipo: Treemap ou barras
        \item Dados: Backend, Frontend, Full Stack, DevOps, Data
        \item Insight: "Backend representa 40\% das vagas"
    \end{itemize}
\end{enumerate}

\textbf{Backend (já existe):}
\begin{verbatim}
GET /api/insights/skills       → Top 10 skills
GET /api/insights/senioridade  → Distribuição por nível
GET /api/insights/modelos      → Presencial/Híbrido/Remoto
GET /api/insights/localizacao  → Top 10 cidades
GET /api/insights/area         → Áreas (Backend, Frontend, etc)
\end{verbatim}

\textbf{O que falta:}
\begin{itemize}
    \item Frontend React consumindo esses endpoints
    \item Gráficos interativos (hover, click to filter)
    \item Filtros por período (últimos 7 dias, 30 dias, etc)
\end{itemize}

%-----------------------------------------------------------------------------

\subsection{Feature 2: Match Inteligente de Vagas}

\textbf{Status:} \textcolor{red}{NÃO IMPLEMENTADO}

\textbf{Objetivo:} Conectar perfil do usuário com vagas compatíveis usando IA.

\textbf{Fluxo do Usuário:}
\begin{enumerate}
    \item Usuário acessa página "Vagas"
    \item Preenche formulário de perfil:
    \begin{itemize}
        \item Skills (multi-select): Java, React, Docker...
        \item Senioridade: Júnior / Pleno / Sênior / Especialista
        \item Modelo preferido: Presencial / Híbrido / Remoto
        \item Localização preferida: (opcional)
        \item Área de interesse: Backend / Frontend / Full Stack / etc
    \end{itemize}
    \item Clica em "Buscar Vagas Compatíveis"
    \item Sistema processa:
    \begin{enumerate}
        \item Backend busca vagas candidatas (filtro no banco)
        \item Envia para IA: perfil + lista de vagas
        \item IA retorna ranking com score de compatibilidade
    \end{enumerate}
    \item Resultado exibido em cards ordenados:
\end{enumerate}

\textbf{Exemplo de Card de Vaga:}
\begin{verbatim}
┌─────────────────────────────────────────────┐
│ 🎯 95% de Compatibilidade                  │
│                                             │
│ Desenvolvedor Full Stack Pleno             │
│ Empresa XYZ Tech | Remoto | São Paulo      │
│                                             │
│ ✅ Match em Skills:                        │
│    Java, Spring Boot, React, Docker        │
│                                             │
│ ❌ Skills que faltam:                      │
│    AWS, Kubernetes                          │
│                                             │
│ 💡 Por que esse match?                     │
│ "Esta vaga combina perfeitamente com seu   │
│  perfil Pleno em Backend. A stack é 80%    │
│  compatível e a empresa aceita Remoto."    │
│                                             │
│ [Ver Detalhes] [Candidatar-se]             │
└─────────────────────────────────────────────┘
\end{verbatim}

\textbf{Backend necessário:}
\begin{verbatim}
POST /api/vagas/match
Body: {
  "skills": ["Java", "Spring Boot", "React"],
  "senioridade": "Pleno",
  "modelo": "Remoto",
  "localizacao": "São Paulo",
  "area": "Backend"
}

Response: {
  "matches": [
    {
      "vagaId": "uuid",
      "titulo": "Dev Full Stack Pleno",
      "empresa": "XYZ Tech",
      "localizacao": "São Paulo",
      "modelo": "Remoto",
      "compatibilidade": 95,
      "skillsMatch": ["Java", "Spring Boot", "React"],
      "skillsFaltam": ["AWS", "Kubernetes"],
      "motivo": "Explicação gerada pela IA...",
      "url": "https://..."
    }
  ],
  "fromCache": false,
  "tokensUtilizados": 450
}
\end{verbatim}

\textbf{Lógica do Backend:}
\begin{enumerate}
    \item \textbf{Pré-filtro no banco (performance)}:
    \begin{itemize}
        \item Buscar vagas que tenham pelo menos 1 skill do usuário
        \item Filtrar por senioridade (exata ou -1 nível)
        \item Filtrar por modelo (se especificado)
        \item Limite: top 50 vagas candidatas
    \end{itemize}

    \item \textbf{Verificar cache}:
    \begin{itemize}
        \item Gerar hash do perfil do usuário
        \item Buscar em \texttt{interacao\_ia} por (tipo=MATCH, hash\_input)
        \item Se existe e é recente (< 24h), retornar do cache
    \end{itemize}

    \item \textbf{Chamar IA}:
    \begin{itemize}
        \item Montar prompt estruturado:
        \begin{itemize}
            \item Perfil do usuário
            \item Lista de 50 vagas candidatas (JSON)
            \item Instrução: ranquear por compatibilidade
        \end{itemize}
        \item IA retorna JSON com ranking
    \end{itemize}

    \item \textbf{Salvar no cache}:
    \begin{itemize}
        \item Persistir em \texttt{interacao\_ia}
        \item Associar hash\_input para futuras consultas
    \end{itemize}

    \item \textbf{Retornar para frontend}:
    \begin{itemize}
        \item JSON com top 10 matches
    \end{itemize}
\end{enumerate}

\textbf{Prompt para IA (exemplo):}
\begin{verbatim}
Você é um especialista em recrutamento tech.

PERFIL DO CANDIDATO:
- Skills: Java, Spring Boot, React, Docker, PostgreSQL
- Senioridade: Pleno
- Modelo preferido: Remoto
- Área: Backend

VAGAS DISPONÍVEIS (JSON):
[
  {
    "id": "123",
    "titulo": "Dev Backend Pleno",
    "empresa": "XYZ",
    "skills": ["Java", "Spring Boot", "AWS", "Kubernetes"],
    "senioridade": "Pleno",
    "modelo": "Remoto"
  },
  ...mais 49 vagas...
]

TAREFA:
Analise e retorne um JSON com os top 10 matches, ordenados
por compatibilidade (0-100). Para cada match, inclua:
- vagaId
- compatibilidade (número 0-100)
- skillsMatch (array de skills que batem)
- skillsFaltam (array de skills que faltam)
- motivo (1 frase explicando o match)

Formato de saída:
{
  "matches": [ ... ]
}
\end{verbatim}

\textbf{Componentes a criar:}
\begin{itemize}
    \item \textbf{Frontend}: Página VagasMatch.jsx
    \item \textbf{Backend}:
    \begin{itemize}
        \item VagaMatchController
        \item VagaMatchService
        \item DTO: MatchRequestDTO, MatchResponseDTO
        \item Query customizada: findCandidateVagas()
    \end{itemize}
\end{itemize}

%-----------------------------------------------------------------------------

\subsection{Feature 3: Roadmap Personalizado de Carreira}

\textbf{Status:} \textcolor{red}{NÃO IMPLEMENTADO}

\textbf{Objetivo:} Gerar plano de estudos estruturado para alcançar objetivo de carreira.

\textbf{Fluxo do Usuário:}
\begin{enumerate}
    \item Acessa página "Meu Roadmap"
    \item Preenche formulário:
    \begin{itemize}
        \item Nível atual: Júnior / Pleno / Sênior / Iniciante
        \item Objetivo: "Desenvolvedor Backend Sênior"
        \item Stack desejada: [Java, Spring Boot, Microserviços, Kafka]
        \item Tempo disponível: "6 meses" / "1 ano" / "2 anos"
        \item Dedicação semanal: "5 horas" / "10 horas" / "20 horas"
    \end{itemize}
    \item Clica em "Gerar Roadmap"
    \item IA processa e retorna roadmap estruturado em fases
    \item Frontend exibe como timeline interativa
\end{enumerate}

\textbf{Visualização do Roadmap (Timeline):}
\begin{verbatim}
════════════════════════════════════════════════════════════
  ROADMAP: Júnior → Desenvolvedor Backend Sênior (6 meses)
════════════════════════════════════════════════════════════

┌─────────────────────────────────────────────────────────┐
│ FASE 1: Fundamentos (Mês 1-2)                          │
├─────────────────────────────────────────────────────────┤
│ Objetivos:                                              │
│  ☐ Dominar Java 17+ (records, sealed classes)         │
│  ☐ Spring Boot 3 (basics)                              │
│  ☐ REST APIs com validação                             │
│  ☐ JPA/Hibernate básico                                │
│                                                         │
│ Recursos:                                               │
│  📚 Curso: "Java Completo" - Udemy                     │
│  📚 Doc oficial: Spring Boot Reference                 │
│  🎥 Vídeo: "REST API Best Practices"                   │
│                                                         │
│ Projeto Prático:                                        │
│  🚀 API de gerenciamento de tarefas (CRUD completo)    │
│     Stack: Java 17, Spring Boot, H2, JPA               │
└─────────────────────────────────────────────────────────┘

┌─────────────────────────────────────────────────────────┐
│ FASE 2: Intermediário (Mês 3-4)                        │
├─────────────────────────────────────────────────────────┤
│ Objetivos:                                              │
│  ☐ PostgreSQL avançado (índices, otimização)          │
│  ☐ Docker e containerização                            │
│  ☐ Testes (JUnit 5, Mockito, TestContainers)          │
│  ☐ Spring Security (JWT)                               │
│                                                         │
│ Projeto Prático:                                        │
│  🚀 E-commerce API com autenticação                    │
└─────────────────────────────────────────────────────────┘

┌─────────────────────────────────────────────────────────┐
│ FASE 3: Avançado (Mês 5-6)                             │
├─────────────────────────────────────────────────────────┤
│ Objetivos:                                              │
│  ☐ Arquitetura de Microserviços                        │
│  ☐ Apache Kafka (produtor/consumidor)                  │
│  ☐ Observabilidade (Prometheus, Grafana)               │
│  ☐ Design Patterns (CQRS, Event Sourcing)             │
│                                                         │
│ Projeto Prático:                                        │
│  🚀 Sistema de pedidos com microserviços e eventos     │
└─────────────────────────────────────────────────────────┘

┌─────────────────────────────────────────────────────────┐
│ CERTIFICAÇÕES RECOMENDADAS                              │
├─────────────────────────────────────────────────────────┤
│  🎓 Oracle Certified Professional: Java SE              │
│  🎓 Spring Professional Certification                   │
│  🎓 AWS Certified Developer - Associate                 │
└─────────────────────────────────────────────────────────┘
\end{verbatim}

\textbf{Backend necessário:}
\begin{verbatim}
POST /api/roadmap/gerar
Body: {
  "nivelAtual": "Júnior",
  "objetivo": "Desenvolvedor Backend Sênior",
  "stackDesejada": ["Java", "Spring Boot", "Microserviços"],
  "tempoDisponivel": "6 meses",
  "horasSemanais": 10
}

Response: {
  "roadmapId": "uuid",
  "titulo": "Júnior → Backend Sênior",
  "duracaoTotal": "6 meses",
  "fases": [
    {
      "numero": 1,
      "nome": "Fundamentos",
      "duracao": "2 meses",
      "objetivos": [...],
      "recursos": [...],
      "projetosPraticos": [...]
    }
  ],
  "certificacoes": [...],
  "fromCache": false
}
\end{verbatim}

\textbf{Lógica do Backend:}
\begin{enumerate}
    \item Verificar cache (hash do input)
    \item Se não existe, chamar IA com prompt estruturado
    \item IA retorna JSON com fases do roadmap
    \item Salvar em \texttt{interacao\_ia}
    \item Retornar para frontend
\end{enumerate}

\textbf{Prompt para IA:}
\begin{verbatim}
Você é um mentor de carreira em tecnologia.

Crie um roadmap detalhado com base em:
- Nível atual: Júnior
- Objetivo: Desenvolvedor Backend Sênior
- Stack desejada: Java, Spring Boot, Microserviços, Kafka
- Tempo disponível: 6 meses
- Dedicação: 10 horas/semana

Estruture em 3-4 fases progressivas. Para cada fase:
- Nome da fase
- Duração estimada
- Lista de objetivos de aprendizado
- Recursos (cursos, livros, vídeos)
- Projetos práticos (com stack sugerida)

Ao final, sugira certificações relevantes.

Retorne um JSON estruturado.
\end{verbatim}

\textbf{Features interativas no Frontend:}
\begin{itemize}
    \item Checkboxes para marcar progresso
    \item Salvar progresso localmente (localStorage)
    \item Exportar como PDF
    \item Compartilhar roadmap (gerar link)
\end{itemize}

\textbf{Componentes a criar:}
\begin{itemize}
    \item \textbf{Frontend}: RoadmapPage.jsx, TimelineComponent.jsx
    \item \textbf{Backend}:
    \begin{itemize}
        \item RoadmapController
        \item RoadmapService
        \item DTO: RoadmapRequestDTO, RoadmapResponseDTO
    \end{itemize}
\end{itemize}

%-----------------------------------------------------------------------------

\subsection{Feature 4: Sugestão de Projetos para Portfólio}

\textbf{Status:} \textcolor{red}{NÃO IMPLEMENTADO}

\textbf{Objetivo:} IA sugere projetos práticos alinhados com gaps de skills do usuário.

\textbf{Fluxo do Usuário:}
\begin{enumerate}
    \item Acessa página "Projetos"
    \item Preenche preferências:
    \begin{itemize}
        \item Tecnologias que domina: [Java, PostgreSQL]
        \item Tecnologias que quer aprender: [Docker, Kafka]
        \item Área de interesse: Backend
        \item Complexidade: Iniciante / Intermediário / Avançado
    \end{itemize}
    \item Clica em "Sugerir Projetos"
    \item IA retorna 3-5 projetos detalhados
\end{enumerate}

\textbf{Exemplo de Projeto Sugerido:}
\begin{verbatim}
┌─────────────────────────────────────────────────────────┐
│ 🚀 Sistema de Notificações em Tempo Real               │
├─────────────────────────────────────────────────────────┤
│ Complexidade: Intermediário                            │
│ Tempo estimado: 3-4 semanas                            │
│                                                         │
│ 📋 Descrição:                                           │
│ Construa um sistema que processa eventos assíncronos   │
│ usando Kafka e envia notificações via WebSocket.       │
│                                                         │
│ 🛠️ Stack sugerida:                                     │
│  • Backend: Java 17, Spring Boot                       │
│  • Mensageria: Apache Kafka                            │
│  • Banco: PostgreSQL                                    │
│  • Infra: Docker, Docker Compose                       │
│                                                         │
│ ✨ Funcionalidades principais:                         │
│  ☐ API REST para criação de eventos                   │
│  ☐ Producer Kafka para publicar eventos               │
│  ☐ Consumer Kafka para processar filas                │
│  ☐ WebSocket para notificações real-time              │
│  ☐ Dashboard para monitorar eventos                    │
│                                                         │
│ 🎯 Diferenciais para portfólio:                        │
│  • Demonstra conhecimento de arquitetura assíncrona   │
│  • Mostra domínio de mensageria                        │
│  • Containerização completa                            │
│  • Pode ser usado em entrevistas técnicas             │
│                                                         │
│ 📚 Recursos para começar:                              │
│  • Tutorial: "Kafka com Spring Boot"                   │
│  • Doc oficial: Apache Kafka Quickstart               │
│                                                         │
│ [Ver Especificação Completa] [Favoritar]               │
└─────────────────────────────────────────────────────────┘
\end{verbatim}

\textbf{Backend necessário:}
\begin{verbatim}
POST /api/projetos/sugerir
Body: {
  "tecnologiasDomina": ["Java", "PostgreSQL"],
  "tecnologiasAprender": ["Docker", "Kafka"],
  "area": "Backend",
  "complexidade": "Intermediário"
}

Response: {
  "projetos": [
    {
      "titulo": "Sistema de Notificações...",
      "descricao": "...",
      "stack": [...],
      "funcionalidades": [...],
      "diferenciais": [...],
      "tempoEstimado": "3-4 semanas",
      "recursos": [...]
    }
  ]
}
\end{verbatim}

\textbf{Lógica do Backend:}
\begin{enumerate}
    \item Verificar cache
    \item Chamar IA com contexto:
    \begin{itemize}
        \item Tecnologias que o usuário domina
        \item Tecnologias que quer aprender
        \item Complexidade desejada
    \end{itemize}
    \item IA gera 3-5 projetos relevantes
    \item Salvar em cache
    \item Retornar para frontend
\end{enumerate}

\textbf{Prompt para IA:}
\begin{verbatim}
Você é um mentor de carreira tech especializado em portfólios.

Sugira 3 projetos práticos baseados em:
- Tecnologias que o usuário domina: Java, PostgreSQL
- Tecnologias que quer aprender: Docker, Kafka
- Área: Backend
- Complexidade: Intermediário

Para cada projeto, retorne:
- Título chamativo
- Descrição (2-3 parágrafos)
- Stack técnica completa
- Lista de funcionalidades (5-7 features)
- Diferenciais para portfólio
- Tempo estimado de desenvolvimento
- Recursos (links, tutoriais)

Os projetos devem:
1. Ser práticos e implementáveis
2. Demonstrar skills valorizadas pelo mercado
3. Ter escopo fechado (não muito amplos)
4. Incorporar as tecnologias que o usuário quer aprender

Retorne JSON estruturado.
\end{verbatim}

\textbf{Features extras:}
\begin{itemize}
    \item Sistema de favoritos (salvar projetos)
    \item Marcar projeto como "Em andamento" ou "Concluído"
    \item Link para GitHub template (se disponível)
    \item Compartilhar projeto com amigos
\end{itemize}

\textbf{Componentes a criar:}
\begin{itemize}
    \item \textbf{Frontend}: ProjetosPage.jsx, ProjetoCard.jsx
    \item \textbf{Backend}:
    \begin{itemize}
        \item ProjetoController
        \item ProjetoService
        \item DTO: ProjetoRequestDTO, ProjetoResponseDTO
    \end{itemize}
\end{itemize}

%=============================================================================
\section{Modelo de Dados}
%=============================================================================

\subsection{Entidades Existentes}

\subsubsection{vaga\_bruta}
Dados brutos coletados pelo scraper.

\textbf{Campos:}
\begin{itemize}
    \item id: UUID (PK)
    \item titulo: VARCHAR(255)
    \item descricao: TEXT
    \item empresa: VARCHAR(255)
    \item localizacao: VARCHAR(255) - nullable
    \item modelo: VARCHAR(25) - Presencial/Híbrido/Remoto
    \item url: VARCHAR(1024) - UNIQUE
    \item palavra\_chave: VARCHAR(25) - termo de busca usado
    \item data\_coleta: TIMESTAMP
    \item processada: BOOLEAN - flag de processamento
\end{itemize}

\subsubsection{vaga\_processada}
Vagas enriquecidas com análise de skills e senioridade.

\textbf{Campos:}
\begin{itemize}
    \item id: UUID (PK)
    \item titulo: VARCHAR(255)
    \item empresa: VARCHAR(255)
    \item senioridade: VARCHAR(100) - Júnior/Pleno/Sênior/Especialista/Estágio
    \item localizacao: VARCHAR(255) - nullable
    \item modelo: VARCHAR(255) - Presencial/Híbrido/Remoto
    \item area: VARCHAR(100) - Backend/Frontend/Full Stack/DevOps/Data
    \item url: VARCHAR(1024)
    \item skills: Many-to-Many com \texttt{skill}
\end{itemize}

\subsubsection{skill}
Catálogo de tecnologias (89+ skills).

\textbf{Campos:}
\begin{itemize}
    \item id: UUID (PK)
    \item nome: VARCHAR(255) - UNIQUE
    \item vagas: Many-to-Many com \texttt{vaga\_processada}
\end{itemize}

\subsubsection{vaga\_skill\_rel}
RelacionamentoMany-to-Many entre vagas e skills.

\textbf{Campos:}
\begin{itemize}
    \item vaga\_id: UUID (FK → vaga\_processada)
    \item skill\_id: UUID (FK → skill)
\end{itemize}

\subsubsection{interacao\_ia}
Armazena todas as interações com IA (para cache e auditoria).

\textbf{Campos:}
\begin{itemize}
    \item id: UUID (PK)
    \item tipo\_funcionalidade: VARCHAR(50) - ROADMAP/PROJETO/MATCH
    \item parametros\_input: JSONB - input do usuário
    \item hash\_input: VARCHAR(64) - SHA-256 do input (para cache)
    \item resposta\_ia: JSONB - output da IA
    \item modelo\_ia: VARCHAR(50) - gemini-1.5-flash / gpt-4o-mini
    \item tokens\_utilizados: INTEGER
    \item data\_criacao: TIMESTAMP
\end{itemize}

\textbf{Índices:}
\begin{itemize}
    \item (tipo\_funcionalidade, hash\_input) - para busca rápida de cache
\end{itemize}

\subsection{Novas Entidades (Opcional para futuro)}

\subsubsection{usuario}
Se quiser adicionar autenticação no futuro.

\textbf{Campos sugeridos:}
\begin{itemize}
    \item id: UUID (PK)
    \item email: VARCHAR(255) - UNIQUE
    \item senha\_hash: VARCHAR(255)
    \item nome: VARCHAR(255)
    \item data\_criacao: TIMESTAMP
\end{itemize}

\subsubsection{perfil\_usuario}
Armazenar perfil do usuário para não precisar preencher sempre.

\textbf{Campos sugeridos:}
\begin{itemize}
    \item id: UUID (PK)
    \item usuario\_id: UUID (FK)
    \item skills: JSONB - array de skills
    \item senioridade: VARCHAR(50)
    \item area\_interesse: VARCHAR(50)
    \item modelo\_preferido: VARCHAR(50)
    \item localizacao: VARCHAR(255)
\end{itemize}

\subsubsection{progresso\_roadmap}
Armazenar progresso do usuário em roadmaps.

\textbf{Campos sugeridos:}
\begin{itemize}
    \item id: UUID (PK)
    \item usuario\_id: UUID (FK)
    \item roadmap\_id: UUID (FK → interacao\_ia)
    \item objetivos\_concluidos: JSONB - array de IDs de objetivos
    \item data\_inicio: TIMESTAMP
    \item data\_ultima\_atualizacao: TIMESTAMP
\end{itemize}

%=============================================================================
\section{APIs REST - Contratos Completos}
%=============================================================================

\subsection{Grupo: Insights (BI)}

\textbf{Status:} \textcolor{green}{IMPLEMENTADO}

\subsubsection{GET /api/insights/skills}
Retorna top 10 tecnologias mais demandadas.

\textbf{Response:}
\begin{verbatim}
[
  {
    "categoria": "Java",
    "totalVagas": 1234
  },
  {
    "categoria": "Python",
    "totalVagas": 987
  }
]
\end{verbatim}

\subsubsection{GET /api/insights/senioridade}
Distribuição de vagas por nível.

\subsubsection{GET /api/insights/modelos}
Distribuição: Presencial/Híbrido/Remoto.

\subsubsection{GET /api/insights/localizacao}
Top 10 cidades com mais vagas.

\subsubsection{GET /api/insights/area}
Distribuição por área (Backend, Frontend, etc).

%-----------------------------------------------------------------------------

\subsection{Grupo: Match de Vagas}

\textbf{Status:} \textcolor{red}{NÃO IMPLEMENTADO}

\subsubsection{POST /api/vagas/match}
Busca vagas compatíveis com perfil do usuário usando IA.

\textbf{Request Body:}
\begin{verbatim}
{
  "skills": ["Java", "Spring Boot", "React", "Docker"],
  "senioridade": "Pleno",
  "modelo": "Remoto",
  "localizacao": "São Paulo",  // opcional
  "area": "Backend"
}
\end{verbatim}

\textbf{Response:}
\begin{verbatim}
{
  "matches": [
    {
      "vagaId": "uuid-123",
      "titulo": "Desenvolvedor Backend Pleno",
      "empresa": "XYZ Tech",
      "localizacao": "São Paulo",
      "modelo": "Remoto",
      "compatibilidade": 95,
      "skillsMatch": ["Java", "Spring Boot", "Docker"],
      "skillsFaltam": ["Kubernetes", "AWS"],
      "motivo": "Stack muito compatível com seu perfil...",
      "url": "https://gupy.io/..."
    }
  ],
  "totalMatches": 10,
  "fromCache": false,
  "tokensUtilizados": 450,
  "modeloIA": "gemini-1.5-flash"
}
\end{verbatim}

\subsubsection{GET /api/vagas/\{id\}}
Buscar detalhes de uma vaga específica.

\textbf{Response:}
\begin{verbatim}
{
  "id": "uuid-123",
  "titulo": "Dev Backend Pleno",
  "empresa": "XYZ Tech",
  "descricao": "Descrição completa...",
  "skills": [
    {"id": "uuid", "nome": "Java"},
    {"id": "uuid", "nome": "Spring Boot"}
  ],
  "senioridade": "Pleno",
  "localizacao": "São Paulo",
  "modelo": "Remoto",
  "area": "Backend",
  "url": "https://..."
}
\end{verbatim}

%-----------------------------------------------------------------------------

\subsection{Grupo: Roadmap}

\textbf{Status:} \textcolor{red}{NÃO IMPLEMENTADO}

\subsubsection{POST /api/roadmap/gerar}
Gera roadmap personalizado usando IA.

\textbf{Request Body:}
\begin{verbatim}
{
  "nivelAtual": "Júnior",
  "objetivo": "Desenvolvedor Backend Sênior",
  "stackDesejada": ["Java", "Spring Boot", "Microserviços"],
  "tempoDisponivel": "6 meses",
  "horasSemanais": 10
}
\end{verbatim}

\textbf{Response:}
\begin{verbatim}
{
  "roadmapId": "uuid-456",
  "titulo": "Júnior → Desenvolvedor Backend Sênior",
  "duracaoTotal": "6 meses",
  "fases": [
    {
      "numero": 1,
      "nome": "Fundamentos",
      "duracao": "2 meses",
      "objetivos": [
        "Dominar Java 17+",
        "Spring Boot básico",
        "REST APIs"
      ],
      "recursos": [
        {
          "tipo": "curso",
          "titulo": "Java Completo",
          "url": "https://udemy.com/..."
        }
      ],
      "projetosPraticos": [
        {
          "titulo": "API de Tarefas",
          "stack": ["Java 17", "Spring Boot", "H2"]
        }
      ]
    }
  ],
  "certificacoes": [
    "Oracle Certified Professional: Java SE",
    "Spring Professional Certification"
  ],
  "fromCache": false,
  "tokensUtilizados": 600
}
\end{verbatim}

%-----------------------------------------------------------------------------

\subsection{Grupo: Projetos}

\textbf{Status:} \textcolor{red}{NÃO IMPLEMENTADO}

\subsubsection{POST /api/projetos/sugerir}
IA sugere projetos para portfólio.

\textbf{Request Body:}
\begin{verbatim}
{
  "tecnologiasDomina": ["Java", "PostgreSQL"],
  "tecnologiasAprender": ["Docker", "Kafka"],
  "area": "Backend",
  "complexidade": "Intermediário"
}
\end{verbatim}

\textbf{Response:}
\begin{verbatim}
{
  "projetos": [
    {
      "titulo": "Sistema de Notificações em Tempo Real",
      "descricao": "Construa um sistema...",
      "stack": ["Java 17", "Spring Boot", "Kafka", "PostgreSQL"],
      "funcionalidades": [
        "API REST para eventos",
        "Producer Kafka",
        "Consumer Kafka",
        "WebSocket notifications"
      ],
      "diferenciais": [
        "Demonstra arquitetura assíncrona",
        "Mostra domínio de mensageria"
      ],
      "tempoEstimado": "3-4 semanas",
      "recursos": [
        {
          "tipo": "tutorial",
          "titulo": "Kafka com Spring Boot",
          "url": "https://..."
        }
      ]
    }
  ],
  "fromCache": false
}
\end{verbatim}

%=============================================================================
\section{Integração com IA Generativa}
%=============================================================================

\subsection{Escolha do Provedor}

\textbf{Opção 1: Google Gemini (Recomendado)}

\textbf{Vantagens:}
\begin{itemize}
    \item Free tier generoso (1500 requests/dia)
    \item Latência baixa
    \item Boa qualidade de output
    \item Suporte a JSON mode
\end{itemize}

\textbf{Modelo sugerido:} gemini-1.5-flash (rápido e barato)

\textbf{Endpoint:}
\begin{verbatim}
POST https://generativelanguage.googleapis.com/v1beta/
     models/gemini-1.5-flash:generateContent?key=API_KEY
\end{verbatim}

%-----------------------------------------------------------------------------

\textbf{Opção 2: OpenAI}

\textbf{Vantagens:}
\begin{itemize}
    \item Qualidade de output superior
    \item Melhor entendimento de contexto
    \item Structured outputs (JSON garantido)
\end{itemize}

\textbf{Desvantagens:}
\begin{itemize}
    \item Pago desde o início
    \item Mais caro que Gemini
\end{itemize}

\textbf{Modelo sugerido:} gpt-4o-mini (custo-benefício)

%-----------------------------------------------------------------------------

\subsection{Estratégia de Cache}

\textbf{Por que cache é crítico:}
\begin{itemize}
    \item Reduz custos com API de IA
    \item Melhora latência (resposta instantânea)
    \item Evita chamadas redundantes
\end{itemize}

\textbf{Implementação:}
\begin{enumerate}
    \item \textbf{Antes de chamar IA:}
    \begin{itemize}
        \item Gerar hash SHA-256 do input (JSON serializado)
        \item Buscar em \texttt{interacao\_ia} por (tipo, hash)
        \item Se encontrado e recente (< 7 dias), retornar do cache
    \end{itemize}

    \item \textbf{Após chamar IA:}
    \begin{itemize}
        \item Salvar resposta em \texttt{interacao\_ia}
        \item Incluir: input, hash, output, tokens, modelo
    \end{itemize}
\end{enumerate}

\textbf{Exemplo de hash:}
\begin{verbatim}
Input JSON:
{
  "funcionalidade": "ROADMAP",
  "nivelAtual": "Júnior",
  "objetivo": "Backend Sênior"
}

Hash SHA-256:
a7f3b2c1d4e5f6g7h8i9j0k1l2m3n4o5p6q7r8s9t0u1v2w3x4y5z6
\end{verbatim}

\textbf{Taxa de hit esperada:}
\begin{itemize}
    \item Roadmaps: 40-50\% (combinações comuns se repetem)
    \item Match: 20-30\% (perfis semelhantes)
    \item Projetos: 30-40\%
\end{itemize}

%-----------------------------------------------------------------------------

\subsection{Rate Limiting}

\textbf{Por que implementar:}
\begin{itemize}
    \item Evitar abuso da API
    \item Controlar custos
    \item Proteger quota do provedor de IA
\end{itemize}

\textbf{Estratégia sugerida:}
\begin{itemize}
    \item \textbf{Por IP}: 10 requests/minuto
    \item \textbf{Por funcionalidade}:
    \begin{itemize}
        \item Match: 5 requests/hora
        \item Roadmap: 3 requests/hora
        \item Projetos: 5 requests/hora
    \end{itemize}
\end{itemize}

\textbf{Implementação:}
\begin{itemize}
    \item Redis com TTL para contadores
    \item Interceptor no Spring Boot
    \item Retornar HTTP 429 (Too Many Requests) quando exceder
\end{itemize}

%-----------------------------------------------------------------------------

\subsection{Monitoramento de Custos}

\textbf{Métricas a rastrear:}
\begin{itemize}
    \item Total de tokens consumidos (por dia/mês)
    \item Custo estimado (tokens × preço do modelo)
    \item Taxa de cache hit
    \item Latência média de chamadas
\end{itemize}

\textbf{Tabela de custos (estimativa):}

\begin{tabular}{|l|l|l|l|}
\hline
\textbf{Modelo} & \textbf{Input (1M tokens)} & \textbf{Output (1M tokens)} & \textbf{Total/request} \\
\hline
gemini-1.5-flash & \$0.075 & \$0.30 & \$0.0002 \\
gpt-4o-mini & \$0.15 & \$0.60 & \$0.0004 \\
\hline
\end{tabular}

\textbf{Estimativa de consumo:}
\begin{itemize}
    \item 1000 usuários/mês
    \item 3 requests/usuário (média)
    \item 50\% cache hit
    \item 1500 chamadas reais
    \item Custo total: \$0.30-0.60/mês (Gemini)
\end{itemize}

%=============================================================================
\section{Roadmap de Implementação}
%=============================================================================

\subsection{Visão Geral das Fases}

\begin{tabular}{|l|l|l|l|}
\hline
\textbf{Fase} & \textbf{Descrição} & \textbf{Duração} & \textbf{Complexidade} \\
\hline
1 & Setup e Infraestrutura IA & 1 semana & Baixa \\
2 & Feature: Match de Vagas & 2 semanas & Alta \\
3 & Feature: Roadmap Personalizado & 1.5 semanas & Média \\
4 & Feature: Projetos Portfólio & 1 semana & Média \\
5 & Frontend: Dashboard + Páginas & 2 semanas & Alta \\
6 & Polimento e Testes & 1 semana & Média \\
\hline
\textbf{TOTAL} & & \textbf{8-9 semanas} & \\
\hline
\end{tabular}

%-----------------------------------------------------------------------------

\subsection{FASE 1: Setup e Infraestrutura IA (1 semana)}

\textbf{Objetivo:} Preparar base para integração com IA.

\textbf{Tarefas Backend:}
\begin{enumerate}
    \item Adicionar dependências no \texttt{pom.xml}:
    \begin{itemize}
        \item spring-boot-starter-webflux (cliente HTTP)
        \item jackson-databind (JSON)
        \item commons-codec (hash SHA-256)
    \end{itemize}

    \item Configurar \texttt{application.properties}:
    \begin{itemize}
        \item API key da IA
        \item Modelo a usar
        \item Parâmetros (temperature, max tokens)
    \end{itemize}

    \item Criar classes base:
    \begin{itemize}
        \item \textbf{GeminiService}: Encapsula chamadas à API Gemini
        \item \textbf{PromptBuilderService}: Constrói prompts estruturados
        \item \textbf{CacheService}: Gerencia cache de respostas IA
        \item \textbf{HashService}: Gera SHA-256 de inputs
    \end{itemize}

    \item Atualizar \texttt{InteracaoIARepository}:
    \begin{itemize}
        \item Método: findByTipoAndHash()
        \item Query customizada
    \end{itemize}

    \item Testar integração básica:
    \begin{itemize}
        \item Criar endpoint de teste: \texttt{POST /api/ia/test}
        \item Enviar prompt simples
        \item Validar resposta
    \end{itemize}
\end{enumerate}

\textbf{Entregas:}
\begin{itemize}
    \item Integração com Gemini funcionando
    \item Cache implementado
    \item Teste básico validado
\end{itemize}

%-----------------------------------------------------------------------------

\subsection{FASE 2: Feature Match de Vagas (2 semanas)}

\textbf{Objetivo:} Implementar matching inteligente entre perfil e vagas.

\textbf{Tarefas Backend (Semana 1):}
\begin{enumerate}
    \item Criar DTOs:
    \begin{itemize}
        \item \textbf{MatchRequestDTO}: Recebe perfil do usuário
        \item \textbf{MatchResponseDTO}: Retorna vagas ranqueadas
        \item \textbf{VagaMatchDTO}: Representa vaga com score
    \end{itemize}

    \item Criar query customizada:
    \begin{itemize}
        \item \textbf{findCandidateVagas()}: Pré-filtro no banco
        \item Busca vagas com pelo menos 1 skill match
        \item Filtra por senioridade e modelo
        \item Retorna top 50 candidatas
    \end{itemize}

    \item Implementar \textbf{VagaMatchService}:
    \begin{itemize}
        \item Orquestra fluxo completo
        \item Valida input
        \item Busca candidatas no banco
        \item Verifica cache
        \item Chama IA se necessário
        \item Salva em cache
        \item Retorna resultado
    \end{itemize}

    \item Criar prompt para IA:
    \begin{itemize}
        \item Template estruturado
        \item Inclui perfil + vagas candidatas
        \item Pede ranking JSON
    \end{itemize}

    \item Implementar \textbf{VagaMatchController}:
    \begin{itemize}
        \item \texttt{POST /api/vagas/match}
        \item Validação de input
        \item Tratamento de erros
    \end{itemize}

    \item Testar com Postman:
    \begin{itemize}
        \item Casos de teste variados
        \item Validar cache funcionando
        \item Verificar persistência
    \end{itemize}
\end{enumerate}

\textbf{Tarefas Frontend (Semana 2):}
\begin{enumerate}
    \item Criar página \textbf{VagasMatch.jsx}:
    \begin{itemize}
        \item Formulário de perfil (skills, senioridade, etc)
        \item Multi-select para skills
        \item Dropdowns para senioridade/modelo
    \end{itemize}

    \item Criar componente \textbf{VagaCard.jsx}:
    \begin{itemize}
        \item Exibe título, empresa, localização
        \item Mostra score de compatibilidade
        \item Lista skills match e gaps
        \item Botão "Ver Detalhes"
    \end{itemize}

    \item Implementar estado:
    \begin{itemize}
        \item useState para formulário
        \item useState para resultados
        \item Loading state
    \end{itemize}

    \item Integrar com API:
    \begin{itemize}
        \item axios.post() para /api/vagas/match
        \item Tratamento de erros
        \item Loading spinner
    \end{itemize}

    \item Estilizar:
    \begin{itemize}
        \item Layout responsivo
        \item Cards bonitos com gradientes
        \item Indicador visual de score (barra/badge)
    \end{itemize}
\end{enumerate}

\textbf{Entregas:}
\begin{itemize}
    \item Backend funcionando e testado
    \item Frontend consumindo API
    \item UX fluida
\end{itemize}

%-----------------------------------------------------------------------------

\subsection{FASE 3: Feature Roadmap Personalizado (1.5 semanas)}

\textbf{Objetivo:} Gerar planos de estudo personalizados.

\textbf{Tarefas Backend (3 dias):}
\begin{enumerate}
    \item Criar DTOs:
    \begin{itemize}
        \item \textbf{RoadmapRequestDTO}
        \item \textbf{RoadmapResponseDTO}
        \item \textbf{FaseRoadmapDTO}
    \end{itemize}

    \item Implementar \textbf{RoadmapService}:
    \begin{itemize}
        \item Lógica de cache
        \item Chamar IA com prompt estruturado
        \item Parsear resposta JSON
    \end{itemize}

    \item Criar prompt para IA:
    \begin{itemize}
        \item Template de roadmap
        \item Pede fases, objetivos, recursos
    \end{itemize}

    \item Implementar \textbf{RoadmapController}:
    \begin{itemize}
        \item \texttt{POST /api/roadmap/gerar}
    \end{itemize}

    \item Testar
\end{enumerate}

\textbf{Tarefas Frontend (5 dias):}
\begin{enumerate}
    \item Criar página \textbf{RoadmapPage.jsx}:
    \begin{itemize}
        \item Formulário de objetivos
        \item Campos: nível atual, objetivo, stack, tempo
    \end{itemize}

    \item Criar componente \textbf{TimelineRoadmap.jsx}:
    \begin{itemize}
        \item Exibe fases em timeline vertical
        \item Checkboxes para objetivos
        \item Accordions para expandir detalhes
    \end{itemize}

    \item Implementar localStorage:
    \begin{itemize}
        \item Salvar progresso localmente
        \item Restaurar ao voltar à página
    \end{itemize}

    \item Integrar com API

    \item Estilizar (timeline bonita com linhas conectoras)
\end{enumerate}

\textbf{Entregas:}
\begin{itemize}
    \item Roadmaps gerados com sucesso
    \item Timeline visual interativa
    \item Progresso persistente
\end{itemize}

%-----------------------------------------------------------------------------

\subsection{FASE 4: Feature Projetos para Portfólio (1 semana)}

\textbf{Objetivo:} Sugerir projetos práticos.

\textbf{Tarefas Backend (2 dias):}
\begin{enumerate}
    \item Criar DTOs
    \item Implementar \textbf{ProjetoService}
    \item Criar prompt para IA
    \item Implementar \textbf{ProjetoController}
    \item Testar
\end{enumerate}

\textbf{Tarefas Frontend (3 dias):}
\begin{enumerate}
    \item Criar página \textbf{ProjetosPage.jsx}
    \item Criar componente \textbf{ProjetoCard.jsx}:
    \begin{itemize}
        \item Card expandível
        \item Lista de funcionalidades
        \item Tags de stack
    \end{itemize}
    \item Integrar com API
    \item Estilizar
\end{enumerate}

\textbf{Entregas:}
\begin{itemize}
    \item Projetos sugeridos com qualidade
    \item Cards visualmente atraentes
\end{itemize}

%-----------------------------------------------------------------------------

\subsection{FASE 5: Frontend - Dashboard e Navegação (2 semanas)}

\textbf{Objetivo:} Criar interface completa e navegação.

\textbf{Tarefas (Semana 1):}
\begin{enumerate}
    \item Setup React:
    \begin{itemize}
        \item \texttt{npx create-react-app frontend}
        \item Instalar dependências (axios, react-router, recharts, tailwind)
    \end{itemize}

    \item Criar estrutura de pastas:
    \begin{itemize}
        \item /components
        \item /pages
        \item /services (API calls)
        \item /utils
    \end{itemize}

    \item Implementar navegação:
    \begin{itemize}
        \item React Router
        \item Navbar com links
        \item Rotas para cada página
    \end{itemize}

    \item Criar página \textbf{DashboardPage.jsx}:
    \begin{itemize}
        \item Consumir endpoints de insights
        \item Gráficos com Recharts
        \item Layout em grid (2 colunas)
    \end{itemize}
\end{enumerate}

\textbf{Tarefas (Semana 2):}
\begin{enumerate}
    \item Estilizar todas as páginas:
    \begin{itemize}
        \item Design system consistente (cores, tipografia)
        \item Responsividade (mobile-first)
        \item Animações sutis (transitions)
    \end{itemize}

    \item Implementar estados globais:
    \begin{itemize}
        \item Context API ou Redux
        \item Loading states
        \item Error handling
    \end{itemize}

    \item Criar componentes reutilizáveis:
    \begin{itemize}
        \item Button
        \item Input
        \item Card
        \item Spinner
    \end{itemize}

    \item Otimizações:
    \begin{itemize}
        \item Code splitting
        \item Lazy loading
        \item Memoization
    \end{itemize}
\end{enumerate}

\textbf{Entregas:}
\begin{itemize}
    \item Aplicação React completa
    \item Todas as páginas funcionais
    \item UX profissional
\end{itemize}

%-----------------------------------------------------------------------------

\subsection{FASE 6: Polimento e Testes (1 semana)}

\textbf{Objetivo:} Garantir qualidade e preparar para produção.

\textbf{Tarefas:}
\begin{enumerate}
    \item \textbf{Testes Backend}:
    \begin{itemize}
        \item Testes unitários (JUnit 5) para Services
        \item Testes de integração para Controllers
        \item Mocks para chamadas de IA
        \item Coverage > 70\%
    \end{itemize}

    \item \textbf{Testes Frontend}:
    \begin{itemize}
        \item Jest para componentes
        \item React Testing Library
        \item Testes de integração (user flows)
    \end{itemize}

    \item \textbf{Performance}:
    \begin{itemize}
        \item Otimizar queries (EXPLAIN ANALYZE)
        \item Adicionar índices se necessário
        \item Configurar pool de conexões
        \item Minificar assets frontend
    \end{itemize}

    \item \textbf{Segurança}:
    \begin{itemize}
        \item Rate limiting implementado
        \item Validação de inputs
        \item Sanitização de dados
        \item CORS configurado corretamente
    \end{itemize}

    \item \textbf{Documentação}:
    \begin{itemize}
        \item README completo
        \item Swagger/OpenAPI para endpoints
        \item Comentários em código complexo
        \item Guia de setup
    \end{itemize}

    \item \textbf{DevOps}:
    \begin{itemize}
        \item Docker Compose atualizado
        \item CI/CD básico (GitHub Actions)
        \item Script de seed de dados
    \end{itemize}
\end{enumerate}

\textbf{Entregas:}
\begin{itemize}
    \item Aplicação testada e otimizada
    \item Documentação completa
    \item Pronto para deploy
\end{itemize}

%=============================================================================
\section{Stack Tecnológica Completa}
%=============================================================================

\subsection{Backend}

\begin{itemize}
    \item \textbf{Framework}: Spring Boot 3.5.7
    \item \textbf{Language}: Java 25
    \item \textbf{ORM}: Spring Data JPA (Hibernate)
    \item \textbf{Database}: PostgreSQL 15
    \item \textbf{Cache}: Redis (Spring Cache)
    \item \textbf{HTTP Client}: WebClient (Spring WebFlux)
    \item \textbf{Migrations}: Flyway 9.22.0
    \item \textbf{Build}: Maven
    \item \textbf{Utilities}: Lombok, Jackson, Apache Commons Codec
\end{itemize}

\subsection{Data Pipeline}

\begin{itemize}
    \item \textbf{Language}: Python 3.10+
    \item \textbf{Web Scraping}: Playwright
    \item \textbf{Database Driver}: psycopg2-binary
    \item \textbf{Text Processing}: unidecode
    \item \textbf{Scheduling}: Cron ou Apache Airflow (futuro)
    \item \textbf{Config}: python-dotenv
\end{itemize}

\subsection{Frontend}

\begin{itemize}
    \item \textbf{Framework}: React 18
    \item \textbf{Language}: JavaScript (ou TypeScript)
    \item \textbf{Routing}: React Router 6
    \item \textbf{HTTP Client}: Axios
    \item \textbf{Charts}: Recharts ou D3.js
    \item \textbf{Styling}: Tailwind CSS ou Material-UI
    \item \textbf{State Management}: Context API ou Redux Toolkit
    \item \textbf{Build}: Vite ou Create React App
\end{itemize}

\subsection{IA Generativa}

\begin{itemize}
    \item \textbf{Provedor}: Google Gemini (primário) ou OpenAI (alternativa)
    \item \textbf{Modelo}: gemini-1.5-flash ou gpt-4o-mini
    \item \textbf{API}: REST via WebClient
\end{itemize}

\subsection{DevOps}

\begin{itemize}
    \item \textbf{Containerização}: Docker + Docker Compose
    \item \textbf{CI/CD}: GitHub Actions
    \item \textbf{Versionamento}: Git + GitHub
    \item \textbf{Deploy}: Render, Railway ou Heroku (opções)
\end{itemize}

%=============================================================================
\section{Pontos de Atenção e Boas Práticas}
%=============================================================================

\subsection{Performance}

\begin{enumerate}
    \item \textbf{Cache é obrigatório}:
    \begin{itemize}
        \item IA é lenta (2-5 segundos/request)
        \item Cache pode reduzir 40-50\% das chamadas
        \item Use Redis para insights + PostgreSQL para IA
    \end{itemize}

    \item \textbf{Pré-filtro no banco}:
    \begin{itemize}
        \item Não envie 10k vagas para IA
        \item Filtre para top 50 candidatas primeiro
        \item Use índices (senioridade, modelo, skills)
    \end{itemize}

    \item \textbf{Paginação}:
    \begin{itemize}
        \item Nunca retorne lista completa
        \item Limite: 10-20 resultados
    \end{itemize}

    \item \textbf{Lazy loading}:
    \begin{itemize}
        \item Carregue skills sob demanda (fetch lazy)
    \end{itemize}
\end{enumerate}

\subsection{Custos com IA}

\begin{enumerate}
    \item \textbf{Preferir Gemini}:
    \begin{itemize}
        \item 1500 requests/dia grátis
        \item Suficiente para MVP
    \end{itemize}

    \item \textbf{Cache agressivo}:
    \begin{itemize}
        \item 50\% de economia fácil
    \end{itemize}

    \item \textbf{Monitorar tokens}:
    \begin{itemize}
        \item Salvar em \texttt{interacao\_ia}
        \item Dashboard de métricas
    \end{itemize}

    \item \textbf{Rate limiting}:
    \begin{itemize}
        \item Evita abuso
        \item Protege orçamento
    \end{itemize}
\end{enumerate}

\subsection{UX}

\begin{enumerate}
    \item \textbf{Loading states}:
    \begin{itemize}
        \item IA demora 2-5s
        \item Mostre spinner + mensagem ("Analisando vagas...")
    \end{itemize}

    \item \textbf{Feedback visual}:
    \begin{itemize}
        \item Score de compatibilidade em cores (verde > 80\%, amarelo 60-80\%, vermelho < 60\%)
    \end{itemize}

    \item \textbf{Empty states}:
    \begin{itemize}
        \item "Nenhuma vaga compatível encontrada"
        \item Sugerir ajustar filtros
    \end{itemize}

    \item \textbf{Erro handling}:
    \begin{itemize}
        \item Mensagens amigáveis
        \item Botão para retry
    \end{itemize}
\end{enumerate}

\subsection{Segurança}

\begin{enumerate}
    \item \textbf{Validação de inputs}:
    \begin{itemize}
        \item Bean Validation no backend
        \item Sanitização de strings
    \end{itemize}

    \item \textbf{Rate limiting}:
    \begin{itemize}
        \item Por IP + por funcionalidade
    \end{itemize}

    \item \textbf{Secrets}:
    \begin{itemize}
        \item API key em variável de ambiente
        \item NUNCA commitar .env
    \end{itemize}

    \item \textbf{CORS}:
    \begin{itemize}
        \item Configurar origins permitidas
    \end{itemize}
\end{enumerate}

\subsection{Qualidade de Código}

\begin{enumerate}
    \item \textbf{Separation of Concerns}:
    \begin{itemize}
        \item Controller: apenas roteamento
        \item Service: lógica de negócio
        \item Repository: acesso a dados
    \end{itemize}

    \item \textbf{DRY (Don't Repeat Yourself)}:
    \begin{itemize}
        \item Extrair lógica comum (CacheService, PromptBuilderService)
    \end{itemize}

    \item \textbf{Error handling consistente}:
    \begin{itemize}
        \item Exception handlers globais
        \item DTOs de erro padronizados
    \end{itemize}

    \item \textbf{Logging}:
    \begin{itemize}
        \item Log chamadas de IA (debug)
        \item Log erros (error)
        \item Estruturado (JSON)
    \end{itemize}
\end{enumerate}

%=============================================================================
\section{Expansões Futuras}
%=============================================================================

\subsection{Features Adicionais}

\begin{enumerate}
    \item \textbf{Autenticação e Usuários}:
    \begin{itemize}
        \item Login/Register (Spring Security + JWT)
        \item Perfil persistente
        \item Histórico de interações
    \end{itemize}

    \item \textbf{Favoritos e Listas}:
    \begin{itemize}
        \item Salvar vagas favoritas
        \item Criar listas personalizadas
        \item Exportar como PDF
    \end{itemize}

    \item \textbf{Notificações}:
    \begin{itemize}
        \item Email quando nova vaga compatível aparecer
        \item Push notifications (PWA)
    \end{itemize}

    \item \textbf{Análises Temporais}:
    \begin{itemize}
        \item Trends de tecnologias (últimos 6 meses)
        \item Crescimento/declínio de skills
        \item Previsão de demanda futura
    \end{itemize}

    \item \textbf{Chat Assistente}:
    \begin{itemize}
        \item Chat genérico para dúvidas
        \item Complementa as páginas principais
    \end{itemize}

    \item \textbf{Integração com LinkedIn}:
    \begin{itemize}
        \item Importar perfil automaticamente
        \item Sugerir melhorias no perfil
    \end{itemize}

    \item \textbf{Múltiplas Fontes de Dados}:
    \begin{itemize}
        \item Scraping de LinkedIn, Indeed, Glassdoor
        \item Agregação de dados
    \end{itemize}

    \item \textbf{Salary Insights}:
    \begin{itemize}
        \item Faixas salariais por stack/senioridade
        \item Comparação de mercado
    \end{itemize}
\end{enumerate}

\subsection{Melhorias Técnicas}

\begin{enumerate}
    \item \textbf{Embeddings para Match}:
    \begin{itemize}
        \item Usar embeddings de IA (mais preciso que IA generativa)
        \item Vector database (Pinecone, Weaviate)
        \item Busca por similaridade
    \end{itemize}

    \item \textbf{Machine Learning Local}:
    \begin{itemize}
        \item Treinar modelo para detectar senioridade
        \item Classificação de área (Backend/Frontend/etc)
        \item Reduz dependência de IA generativa
    \end{itemize}

    \item \textbf{GraphQL}:
    \begin{itemize}
        \item Alternativa ao REST
        \item Queries mais flexíveis
    \end{itemize}

    \item \textbf{Real-time Updates}:
    \begin{itemize}
        \item WebSockets para notificações
        \item Server-Sent Events para dashboard
    \end{itemize}

    \item \textbf{Mobile App}:
    \begin{itemize}
        \item React Native
        \item Compartilhar lógica com web
    \end{itemize}
\end{enumerate}

%=============================================================================
\section{Conclusão}
%=============================================================================

\subsection{Resumo Executivo}

\textbf{Tech Career Pulse} é uma plataforma completa de inteligência de mercado tech que:

\begin{itemize}
    \item \textbf{Coleta} dados reais de vagas via web scraping
    \item \textbf{Processa} e enriquece com NLP (skills, senioridade)
    \item \textbf{Analisa} tendências e gera insights visuais
    \item \textbf{Utiliza IA} contextualmente em 3 funcionalidades:
    \begin{enumerate}
        \item Match inteligente de vagas
        \item Roadmaps de carreira personalizados
        \item Sugestões de projetos para portfólio
    \end{enumerate}
\end{itemize}

\subsection{Diferenciais}

\begin{enumerate}
    \item \textbf{Dados reais}: Não é chatbot genérico, usa vagas reais
    \item \textbf{IA contextual}: Integrada em workflows, não chat isolado
    \item \textbf{Full-stack showcase}: Demonstra habilidades completas
    \item \textbf{Arquitetura robusta}: Padrões profissionais (MVC, cache, migrations)
    \item \textbf{UX profissional}: Páginas dedicadas, não chat linear
\end{enumerate}

\subsection{Métricas de Sucesso}

\textbf{Para portfólio:}
\begin{itemize}
    \item Projeto complexo e bem documentado
    \item Múltiplas tecnologias integradas
    \item Use cases reais e práticos
    \item Código limpo e testado
\end{itemize}

\textbf{Para usuários:}
\begin{itemize}
    \item > 1000 vagas no banco
    \item Taxa de match > 80\% de satisfação
    \item Roadmaps acionáveis e práticos
    \item Tempo de resposta < 3 segundos
\end{itemize}

\subsection{Próximos Passos}

\begin{enumerate}
    \item Revisar este documento
    \item Configurar ambiente de desenvolvimento
    \item Seguir roadmap de implementação (8-9 semanas)
    \item Testar incrementalmente
    \item Deploy em produção
    \item Documentar no GitHub
    \item Adicionar ao portfólio
\end{enumerate}

\subsection{Recursos Úteis}

\begin{itemize}
    \item \textbf{Gemini API}: \url{https://ai.google.dev/gemini-api}
    \item \textbf{Spring Boot Docs}: \url{https://spring.io/projects/spring-boot}
    \item \textbf{React Docs}: \url{https://react.dev}
    \item \textbf{Recharts}: \url{https://recharts.org}
    \item \textbf{Tailwind CSS}: \url{https://tailwindcss.com}
\end{itemize}

\vspace{1cm}

\begin{center}
\textit{Documento gerado em \today}

\textit{Tech Career Pulse - Versão 1.0}
\end{center}

\end{document}
